\documentclass{article}
\usepackage[utf8]{inputenc}
\usepackage[ngerman]{babel}
\usepackage{mathtools} 
\usepackage[T1]{fontenc}
\usepackage{amsmath}
\usepackage{amssymb}
\usepackage{amsthm}
\title{Proof 10=9.99999...}
\author{Johannes Steppe}
\date{December 2017}

\DeclarePairedDelimiter{\abs}{\lvert}{\rvert} 
\DeclarePairedDelimiter{\norm}{\lVert}{\rVert} 

\begin{document}

\maketitle

\section{Introduction}
In the following Proof I will show, that \[9.\overline{9}=10.\]
\newline
\newline As we know from the geometric series:
for    \(     \abs{x} <1\) 

\[
{\displaystyle \sum_{i=0}^{\infty}}{x^i}=\frac{1}{1-x}
\]
We will need this later.
\section{Proof}



let  $M=9.\overline{9}$
\[M=9+\frac{9}{10}+\frac{9}{100}+\frac{9}{1000}...\]

so when we divide by 9\[\frac{1}{9}M=1+\frac{1}{10}+\frac{1}{100}+\frac{1}{1000}...\]

and now we can write the infinite addition as a geometric series\[
\frac{M}{9}=\sum_{i=0}^{\infty}{\left(\frac{1}{10}\right)^i}
\]
so as we can see from the introduction
\[\frac{M}{9}=\frac{1}{1-\frac{1}{10}}=\frac{10}{9}\]
so when we multiply by 9\[M=10=9.\overline{9}\]
Q.E.D.



\end{document}







}